% Introduction
%
%	Motivation
%	State of the Research
%	Goals and Outline

\chapter{Introduction}

In a particle accelerator, the charged particles circulate around the ring
and oscillate due to the magnets and the accelerating structures. The 
accelerating structures, in the \gls{LHC} supra-conducting \glspl{cavity} 
apply a strong electrical field that oscillates at the \gls{rffreq} 
to particles in order to collect and accelerate particles in \glspl{bunch} 
inside a frequency \gls{bucket}.

The particles inside a bucket are oscillating longitudinally along the 
ring and transversally in the vertical and horizontal plane. The longitudinal
oscillations are damped by the beam control system. But the transversal
oscillations must be damped by a separate system~: the 
\gls{ADT}\cite{Benews11,Zhabitsky:1141925}.

One of the key parameters of the accelerator is the betatron tune. The 
betatron tune, $Q$, is the quotient of the betatron oscillation and the 
particle frequencies.

$$f_\beta = Q * f_0$$

This value allow us to check if the particle beam is stable and don't reach
any dangerous instabilities.

\section{Tune measurement in the LHC}

In order to measure the betatron tune in an accelerator we use \glspl{BPM}.
These monitors are able to measure the position of the beam in the vacuum 
chamber.

\subsection{Actual system}

In the present setup \gls{BI} group is using their \gls{BBQ} 
\cite{Boccardi:1156349} system to acquire the tune over a certain number
of machine turn (256 to 128'000). This can work as a passive instrument or as 
an on demand system by exciting 12 \glspl{bunch} in the beam with the 
\gls{MKQA}. \Gls{ADT} has also been used for tune measurement 
excitation\cite{HofleEvian10}. 

In normal operation, as the \gls{ADT} is active, it is difficult to have a
good picture of the excited bunches and make a fine tune measurement~: the
oscillations created by the \gls{MKQA} are damped by the \gls{ADT}. There
have been studies to disable the \gls{ADT} for a certain number of bunches in
order to get a better tune measurement\cite{HofleEvian11}, but this may
not be sufficient.

\subsection{Proposed system}

The \gls{ADT} also have \glspl{BPM} and these can have per bunch 
measurement\cite{BphMeas07}. This could allow a much precise measurement. But due 
to the high among of data to be processed (estimated to 640 mega bytes per 
seconds for each \gls{BPM}) we need dedicated hardware to compute the 
correct tune\cite{HofleChamonix12}.

During the 2012 normal operation of the \gls{LHC}, data will be acquired using
the \gls{ADT} acquisition system and data processing techniques will be 
tried to asses the modification that will be needed in order to make a 
reliable \gls{tune} measurement at a reasonable rate\cite{HofleChamonix12}.

The current \gls{VME} implementation has some serious issues in particular 
the bus is quite slow the data rate of the bus is around 40 megabits per 
seconds. The data need to either be processed on the acquisition board or 
to be off-loaded to another computer using the serial link available on the 
board\cite{Baudrenghien:1124094}.

\subsubsection{DSP on VME board}

\Glspl{DSP} are able to compute \glspl{FFT} at high rate and these are used
already in the machine at different places to make high speed feedback loops.
The question is~: is it fast enough to compute all the \glspl{FFT} needed, 
\glspl{DSP} are two orders of magnitude slower than \glspl{GPU}. We also 
would have to develop a completely new system in order to be able to use them, 
in fact we don't have \glspl{DSP} in the present \gls{ADT}. The cost of 
development and the complexity of the deployment should also be studied.

\subsubsection{FPGA pre-processing on VME board}

Like in the approach using \glspl{DSP} on VME boards, the question of computing 
power is still unsolved. We already have in house experience and we already 
have a lot of \gls{FPGA} installed in the \gls{ADT}. But if we want to do it we 
will have to create a new card able to replace the existing one and to make the 
computation. This mean create a potential problem in the existing setup. The 
cost is also to be studied we have to develop a new card, test it and install 
it in the \gls{LHC}.

\subsubsection{GPU off-board computing}

This solution can be integrated easily in the present setup. The present 
acquisition cards already have a digital output and could be used to transfer
the data in another crate that could do the computations. The \glspl{GPU} are
cheap (compare to the price of developing a new \gls{VME} card) and easily
scalable. The \gls{GPU} should have largely enough computing power to be able
to make the \glspl{FFT}. Another interesting aspect of this solution is the 
ability to test it using \gls{CPU}.
