% Graphic Processing Unit
%
%	Take from the PA

\chapter{Graphic Processing Unit}

\section{Introduction}

General purpose programming on graphical processors is a new field with a growing community of practitioners. Until recently only proprietary interfaces existed to harness the power of these chips. With the arrival of the \gls{OpenCL} a new open interface has appeared, and with it a hope for a unified, simple and portable framework for general purpose computing on heterogeneous hardware.

\section{OpenCL a quick overview}

The \gls{OpenCL} is the open standard for \gls{GPGPU} programming. It is maintained by the Khronos group, and was initially proposed by Apple. Many companies of the industry are members of the \gls{OpenCL} Working group: Altera, AMD, Apple, ARM, Broadcom, Codeplay, DMP, EA, Ericsson, Fixstars, Freescale, Hi corp, IBM, Intel, Imagination Technologies, Kestrel Institute, Kishonti, Los Alamos, Motorola, Movidius, Multicoreware, Nokia, NVIDIA, OpenEye, Presagis, Qualcomm, Rightware, Samsung, ST, Symbio, Texas Instruments, The University of West Australia, Vivante and Xilinx.
\begin{quotation}
OpenCL\texttrademark is the first open, royalty-free standard for cross-platform, parallel programming of modern processors found in personal computers, servers and handheld/embedded devices. \gls{OpenCL} greatly improves speed and responsiveness for a wide spectrum of applications in numerous market categories from gaming and entertainment to scientific and medical software.
\end{quotation}
\begin{quotation}
The Khronos Group is a not for profit industry consortium creating open standards for the authoring and acceleration of parallel computing, graphics, dynamic media, computer vision and sensor processing on a wide variety of platforms and devices. All Khronos members are able to contribute to the development of Khronos \gls{API} specifications, are empowered to vote at various stages before public deployment, and are able to accelerate the delivery of their cutting-edge 3D platforms and applications through early access to specification drafts and conformance tests.
\end{quotation}

\subsection{Architecture}

