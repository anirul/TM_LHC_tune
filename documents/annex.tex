% Methodology
%
%	Experimental Set-up
%	Model Equations

\chapter{Annex}

\section{Estimation of the amount of data}

Presently the \gls{LHC} is working with an interval of 50~ns between \glspl{bunch}. This correspond to a bunch every 20 \glspl{bucket}. But the \gls{OP} is planning to move to 25ns \glspl{bunch} spacing this would mean 5 \glspl{bucket} between \glspl{bunch}. With the \gls{rffreq} we can compute the number of acquisitions per seconds.

$${\rm for~}50{\rm ~ns}~: \frac{400.789*10^{6}}{20} = 20\;039\;450 \leq 2^{25}$$

$${\rm for~}25{\rm ~ns}~: \frac{400.789*10^{6}}{10} = 40\;078\;900 \leq 2^{26}$$ 

This represent the amount of data for one pickup (\gls{BPM}), in the case of \gls{ADT} we have two of them per beam and per plane so as the \gls{LHC} has two rings and for each ring there are two transversal plane and there are two pickups per plane. This means we still have to multiply this value by eight.

$${\rm for~}50{\rm ~ns}~: 2^{25} * 8 = 2^{28}$$

$${\rm for~}25{\rm ~ns}~: 2^{26} * 8 = 2^{29}$$

As \glspl{FFT} on \glspl{GPU} start to be be faster than \glspl{CPU} around $2^{15}$ acquisitions it seems interesting to study this kind of system to compute the \gls{tune}.

\section{Measurement with the ADT}

In order to check the feasibility of the system and to have a good prototype the first test will be to excite some of the \glspl{bunch} and acquire the \gls{tune} using the \gls{ADT} during the end of 2012 run\cite{Valuch12}.

A piece of software has been developed that will acquire the bunch by bunch acquisition and compute various algorithm on the data using the \gls{CPU} and the \gls{FFTW} library in the \gls{CERN} infrastructure using \gls{CO} group control system and the \gls{OP} group infrastructure.

\section{Experimental Set-up}

   \subsection{Hardware}

	The experimental set up is not presently able to acquire more than a certain number of bunches due to memory limitation 16k and interupt frequency so during the \glspl{MD} only 6 bunches were acquired by bunch by planes.

   \subsection{Software}

\section{FFT}

Used the algorithm described here\cite{Govindaraju07}.

\section{SVD}

Used the GNU scientific library.

\section{Machine development sessions}

Using the \gls{ADT} \glspl{BPM} we acquired data in the machine during 3 independant \glspl{MD}. Most of the data taking was done in parallel to other normal LHC operation or during \gls{ADT} dedicated \gls{MD} time.

   \subsection{First session}
   
   Night session of the 11 october 2012.

   \subsection{Second session}
   
   Parasitic session of the 16 october 2012

   \subsection{Third session}

   Ramp acquisition of the 14 november 2012

