%my glossary (for the TM LHC tune)

\newglossaryentry{bunch}{
   name=bunch,
   description={
      particules trapped inside an \gls{RF} bucket circulating in the 
      machine},
   plural=bunches}
\newglossaryentry{bucket}{
   name={bucket},
   description={
      at every \gls{RF} period in the \gls{rffreq} there is a 
      bucket, in each of these bucket a particles bunch can potentially be 
      stored in the ring},
   plural={buckets}}
\newglossaryentry{rffreq}{
   name={RF frequency ($f_0$)},
   description={
      the base frequency of the \gls{RF} in the \glspl{cavity} in the case of
      the LHC this frequency is 400.789MHz, this frequency dictate the number of
      bucket that the machine can have}}
\newglossaryentry{cavity}{
   name={cavity},
   description={
      \gls{RF} structure made to accelerate the particles, it uses a high power 
      radio frequency into a resonating structure to increase the energy 
      of the particles},
   plural={cavities}}
\newglossaryentry{VME}{
   name={VMEbus},
   description={
      a computer bus standard widespread at CERN, in the case of the LHC
      \gls{RF} the bus has a larger board and some of the pins are used to 
      route custom signals between cards}}
\newglossaryentry{kicker}{
   name={kicker},
   description={
      machine in an accelerator that can kick the beam transversally, used to
      kick the beam in or out (injection or extraction kicker) of the beam pipe 
      but also in our case excite the beam transversally},
   plural={kickers}}
\newglossaryentry{damper}{
   name={damper},
   description={
      machine in an accelerator that damp the transverse oscillation of the 
      beam by applying a transverse electric field},
   plural={dampers}}
\newglossaryentry{tune}{
   name={betatron tune},
   description={
      the betatron tune is the frequency of the oscillations of the 
      \glspl{bunch} divided by the \gls{rffreq}}}
\newglossaryentry{h}{
   name={harmonic number ($H$)},
   description={
      the harmonic number is the number of bucket an accelerator can have in 
      the ring, in the case of the \gls{LHC} 35'640}}
\newglossaryentry{FFTW}{
   name={FFTW},
   description={
      is a C subroutine library for computing the discrete Fourier transform 
      (DFT) in one or more dimensions, of arbitrary input size, and of both 
      real and complex data (as well as of even/odd data, i.e. the discrete 
      cosine/sine transforms or DCT/DST)}}

