% conclusion

\chapter{Conclusion}

\glsreset{GPU}
\glsreset{FFT}
\glsreset{LHC}

The betatron \gls{tune} is a critical parameter of a particle collider machine. In the \gls{LHC} the \gls{tune} acquisition is computed by doing \glspl{FFT} on an averaged value over all the bunches. In order to have a decent on-line \gls{FFT} computation of all individual bunches a faster method is needed. 

\Gls{GPU} used in graphic card are faster than normal \gls{CPU} for
parallelized computations. Using a \gls{GPU} we can accelerate
\gls{FFT} computation by a factor of 10 especially if we have
multiples \glspl{FFT} to be computed at the same time. This could,
acording to the timing asked by the operation, allow for a new on-line
acquisition of the \gls{tune}.

New hardware will have to be deployed in the \gls{ADT} Faraday cage. This will in turn need new software to drive it. But is should be accessible as most of it is already present in the \gls{ADT} setup.

On-line computation of \glspl{FFT} for all the individual bunch is shown to be possible this will allow a better bunch-by-bunch surveillance of the beam and better statistic on what is causing transversal instabilities. This could allow to have a better tune acquisition and allow for a better beam time life in the \gls{LHC}.