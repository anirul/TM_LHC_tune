% conclusion

\glsresetall
\chapter{Conclusion}

The \gls{tune} is a critical parameter of a particle collider machine. In the \gls{LHC} the \gls{tune} acquisition is currently computed by doing \glspl{FFT} on an averaged value over all the bunches. In order to have a useful on-line \gls{FFT} computation of all individual bunches a faster method is needed. 

\Glspl{GPU} used in graphics cards are faster than normal \glspl{CPU} for parallelized computations. Using a \gls{GPU} we can accelerate \gls{FFT} computation by a factor of 10 especially if we have multiple \glspl{FFT} to be computed at the same time. This could allow for a new on-line bunch-by-bunch acquisition of the \gls{tune} with an update rate sufficient for the LHC operation requirements.

New hardware will have to be deployed in the \gls{ADT} signal processing electronics, which will in turn need new software to drive it. But this should be achievable as most of it is already present in the \gls{ADT} setup.

On-line computation of \glspl{FFT} for all the individual bunches is shown to be possible. This will allow a better bunch-by-bunch surveillance of the beam and better statistics on what is causing transverse instabilities. This could allow us to have a better tune acquisition and allow for a better beam time life in the \gls{LHC}.
