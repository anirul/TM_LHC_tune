% Appendix
%
%	Experimental Set-up
%	Model Equations

\chapter{Appendix}

More in depth view of the software and data acquisition.

\section{Estimation of the amount of data}

Presently the \gls{LHC} is working with an interval of 50~ns between \glspl{bunch}. This correspond to a bunch every 20 \glspl{bucket}. But the \gls{OP} is planning to move to 25ns \glspl{bunch} spacing this would mean 5 \glspl{bucket} between \glspl{bunch}. With the \gls{rffreq} we can compute the number of acquisitions per seconds.

$${\rm for~}50{\rm ~ns}~: \frac{400.789*10^{6}}{20} = 20\;039\;450 \leq 2^{25}$$

$${\rm for~}25{\rm ~ns}~: \frac{400.789*10^{6}}{10} = 40\;078\;900 \leq 2^{26}$$ 

This represent the amount of data for one pickup (\gls{BPM}), in the case of \gls{ADT} we have two of them per beam and per plane so as the \gls{LHC} has two rings and for each ring there are two transversal plane and there are two pickups per plane. This means we still have to multiply this value by eight.

$${\rm for~}50{\rm ~ns}~: 2^{25} * 8 = 2^{28}$$

$${\rm for~}25{\rm ~ns}~: 2^{26} * 8 = 2^{29}$$

As \glspl{FFT} on \glspl{GPU} start to be be faster than \glspl{CPU} around $2^{15}$ acquisitions it seems interesting to study this kind of system to compute the \gls{tune}.

\section{Measurement with the ADT}

In order to check the feasibility of the system and to have a good prototype the first test will be to excite some of the \glspl{bunch} and acquire the \gls{tune} using the \gls{ADT} during the end of 2012 run\cite{Valuch12}.

A piece of software has been developed that will acquire the bunch by bunch acquisition and compute various algorithm on the data using the \gls{CPU} and the \gls{FFTW} library in the \gls{CERN} infrastructure using \gls{CO} group control system and the \gls{OP} group infrastructure.

\section{Experimental Set-up}

This is a basic description of the setup used during the \gls{MD} of 2012 during witch some data acquisition was made in the \gls{LHC}. The low level analog part of the system is the same than the system that will be used in the future system, but some part described were not present at the moment of the data acquisitions.

\subsection{Hardware}

The experimental set up is not presently able to acquire more than a certain number of bunches due to memory limitation 16k and interrupt frequency so during the \glspl{MD} only 6 bunches were acquired by bunch by planes.

\subsection{Software}

The software used during data acquisition was only able to get the data from the card and copy it on a disk that was available on the \gls{CERN} \gls{NFS}. The software was able to compute the \gls{FFT} but only using \gls{FFTW} and no \gls{GPU}.

\section{FFT}

Used the algorithm described here\cite{Govindaraju07}. The OpenCL Kernel is described here and composed of two part the twiddle and the main radix-2 part.

\begin{figure}[H]
\centering
\caption{OpenCL kernel code for the twiddle (used by FFT)}
\label{fig:twidle_cl}
\begin{lstlisting}
float2 twiddle(float2 a, int k, float alpha)
{
	float cs,sn;
	sn = sincos((float)k * alpha, &cs);
	return mul(a, (float2)(cs, sn));
}
\end{lstlisting}
\end{figure}

There is a difference between the \gls{FFT} described in the algorithm cited above and the one you can see here. Here the operation is parallized to be made on multiples acquisitions at the same time, as in typical operation we have to compute 2880 FFT at the same time. So the thread is multiply by two factor the size of the acquisition and the number of \glspl{FFT} to be made.

\begin{figure}[H]
\centering
\caption{OpenCL kernel code for multiple FFTs}
\label{fig:fft_cl}
\begin{lstlisting}
__kernel void fftRadix2Kernel(
	__global const float2 * x,
	__global float2 * y,
	const int p)
{
	// thread count
	int t = get_global_size(0);
	// thread index
  	int i = get_global_id(0);
	// fft index
  	int z = get_global_id(1);
  	// index in input sequence, in 0..P-1
  	int k = i & (p - 1);
  	// output index
  	int j = ((i - k) << 1) + k;
  	float alpha = -3.14159265359f * (float)k / (float)p;
  	
	// Read and twiddle input
	x += z * t * 2;
	x += i;
	float2 u0 = x[0];
	float2 u1 = twiddle(x[t], 1, alpha);
	
	// In-place DFT-2
	float2 tmp = u0 - u1; 
	u0 += u1; 
	u1 = tmp;
	
	// Write output
	y += z * t * 2;
	y += j;
	y[0] = u0;
	y[p] = u1;
}
\end{lstlisting}
\end{figure}

It is clear that the parallelization has a maximum equal in our case to half the number of samples $N$ / 2 multiplied by the number of \glspl{FFT}. On our test this mean 1024 * 2880 far more than the number of core we can find on a top level \gls{GPU} as you can see on table~\ref{tab:kepler}.

\section{SVD}

Used the \gls{GSL} to compute SVD from the data I had. This mean that technically the ``svd'' software should be under \gls{GPL}.

\Gls{GSL} offer different method for calculating the SVD of matrices, but the computing is only available in double precision float. The different approaches have different result varying with the size and shape of the matrix.

\section{Machine development sessions}

Using the \gls{ADT} \glspl{BPM} we acquired data in the machine during 3 independent \glspl{MD}. Most of the data taking was done in parallel to other normal LHC operation or during \gls{ADT} dedicated \gls{MD} time.

\subsection{First session}

Night session of the 11 October 2012. 

This was the first session and the software still had some issues. We experienced some difficulties in injecting in the \gls{LHC} that were not really related to us.

Overall we got some nice result and that was very encouraging for further test.

\subsection{Second session}

Parasitic session of the 16 October 2012. 

The bugs were fixed and we could have a short acquisition session that happen in parallel with other operation.

This session allowed us to validate the acquisition software in normal operation and with the changes that where made in between the two \gls{MD}.

\subsection{Third session}

Ramp acquisition of the 14 November 2012. 

During this afternoon we had some machine time to acquire the tune with the acquisition software on low gain bunches during ramp and collision.

The software worked well but the noise at the end of the ramp was too big and the tune was not obviously visible.

\section{Source code}

The source code is available online for the data analysis software under GIT and inside the \gls{CERN} infrastructure for the \gls{FESA} classes under CVS. Code published by \gls{CERN} is by default under \gls{GPL} v3.

\subsection{FESA ADTDSPU}

The source files for the \gls{ADTDSPU} \gls{FESA} class is available on \gls{CERN} central \gls{CVS} servers at~: \url{http://isscvs.cern.ch/cgi-bin/viewcvs-all.cgi/ALLADTDSPU/v210/?root=FESA-equipment}.

\subsection{FESA Tune Acquisition}

The source files for the tune acquisition software is also available on \gls{CERN} central \gls{CVS} servers at~: \url{http://isscvs.cern.ch/cgi-bin/viewcvs-all.cgi/ALLADTTuneMeas/v0/?root=FESA-equipment}.

\subsection{Data analysis}

The source files for the data analysis software are available on GitHub at~: \url{https://github.com/anirul/TM_LHC_tune} and is under BSD type license.
